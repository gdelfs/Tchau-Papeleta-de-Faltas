\chapter{Conclusão}
\noindent
Este trabalho teve como objetivo desenvolver um sistema informatizado, que utilizasse o reconhecimento facial em conjunto com a tecnologia de desenvolvimento móvel, capaz de ser uma alternativa para o atual sistema de apuração de faltas dos alunos do IME. Após o estudo conceitual sobre métodos de detecção e reconhecimento facial, foi possível  entender melhor as possibilidades oferecidas pela utilização dos diferentes algoritmos pesquisados. Tal entendimento tornou possível a escolha do algoritmo que melhor atenderia ao objetivo do projeto: o LBPH.

O sistema desenvolvido integra uma aplicação \textit{Android}, um banco de dados, um servidor de reconhecimento facial e  um serviço \textit{web} para a realização da apuração de faltas dos alunos do IME. Apesar de trabalharem para o mesmo objetivo, esses elementos são independentes entre si, de forma que qualquer um deles pode ser modificado ou substituído sem afetar o sistema como um todo. Dessa forma, a manutenibilidade do sistema fica facilitada permitindo que futuros projetos deem continuidade ao estudo e proponham soluções de melhoria a partir do aprimoramento dos módulos conforme a necessidade ou o surgimento de técnicas novas.

Por fim, o resultado alcançado mostrou-se compatível ao objetivo proposto, apresentando uma interface amigável e de rápido entendimento para os usuários . Além disso, funções como a correção da papeleta, realizada pelo serviço \textit{web}, onde o professor pode alterar manualmente a presença de um aluno, ajudam a aumentar a confiabilidade do sistema, uma vez que o reconhecimento facial pode falhar. %Outra função que merece destaque é o armazenamento da fotografia analisada, uma vez que permite a verificação da presença \textit{a posteriori}, caso o aluno conteste sua falta. 

%FUNDAMENTAR ESCOLHA
%Optou-se pelo desenvolvimento em módulos buscando dividir o problema em partes e facilitar a eventual substituição pontual de determinado módulo. Na possibilidade de obter-se um algoritmo de reconhecimento facial com melhor qualidade no futuro, ou seja, com taxas de acerto mais elevadas, o módulo Servidor seria passível de substituição pelo novo a um custo reduzido, por exemplo.

%Por fim, o fato do projeto ter sido desenvolvido de maneira modular permite que futuros projetos deem continuidade ao estudo e proponham soluções de melhoria sem que, para isso, percam a oportunidade de utilizar parte daquilo que já foi desenvolvido.

%No módulo Servidor, escolheu-se o uso da linguagem Python pela sua facilidade de leitura, compreensão e manutenção. Além disso optou-se pela biblioteca OpenCV, que apresenta a implementação do algortimo LBPH. Com base nos testes de reconhecimento facial, utilizando o banco de imagens disponibilizado pela Universidade de Yale \citep{yales}  apresentados no capítulo 4, escolheu-se os parâmetros para o algoritmo LBPH de maneira a obter a maior taxa de acerto possível. 

%No módulo \textit{Mobile}, optou-se pelo sistema Android, desenvolvendo um aplicativo de uso intuitivo. Como abordado ao longo do capítulo 5, a escolha pelo desenvolvimento em Android contribui para a tentativa de tornar o aplicativo tão acessível quanto possível e aumenta a qualidade do produto final pela possibilidade de usar componentes já prontos intrínsecos ao sistema operacional. A linguagem utilizada no desenvolvimento foi a recomendada pela documentação oficial, Java.\citep{Android1} Para manter a coerência de linguagem dentro do módulo, utilizou-se a mesma linguagem na construção do servidor TCP que receberá toda a informação. A escolha pela utilização de socket TCP/IP, também embasada ao longo do capítulo 5, deu-se fundamentalmente pela verificação de envio correto das informações ao utilizar tal protocolo.

%Conluída esta etapa, já é possível que um professor, utilizando o aplicativo, preencha os dados da aula, selecione uma imagem, da galeria ou câmera fotográfica, e envie o pedido de apuração ao módulo Servidor. Dar-se-á, então, inicio à detecção de faces na fotografia  selecionada, reconhecendo cada uma dessas com base no arquivo de treinamento para a turma em questão. Ao final, as informações de quais alunos tiveram sua face detectada serão exibidas na tela do servidor, adicionando-se também a matrícula dos alunos em cada uma das faces da fotografia enviada.

%Após o estudo conceitual sobre métodos de detecção e reconhecimento facial, foi possível  entender melhor as possibilidades oferecidas pela utilização dos diferentes algoritmos pesquisados. Tal entendimento tornou possível a escolha daquele  algoritmo que melhor atenderia os objetivos propostos: o LBPH.

%A escolha foi feita principalmente devido ao baixo valor de distância entre as fotos analisadas durante os testes, ou seja, na certeza obtida na detecção de cada uma das faces em teste. Considerou-se essa característica como fundamental pois é imprescindível ou determinar com precisão quem é o aluno presente ou reconhecer que não se tem essa informação com uma precisão satisfatória. Usando então valores adequados para o parâmetro de distância máxima permitida, \textit{threshold}, busca-se evitar dar presença para um aluno que não está em sala ao invés de se dar presença para aquele aluno que estava em sala, i.e. confundir dois alunos. 

%Além disso, ficou claro que o processamento de detecção e o arquivo de inteligência necessário deveriam ser mantidos em um servidor dada a necessidade de um maior poder de processamento do que aqueles comumente oferecidos por aparelhos mobile de baixo custo para efetuar o processo de reconhecimento.

%Considerou-se ainda durante o desenvolvimento a possibilidade de, no futuro, melhorar a qualidade do algoritmo de reconhecimento utilizado. Devido a isso, ele foi colocado em um módulo à parte, sendo facilmente substituível em uma eventual necessidade, visto que implementar um algoritmo com eficiência superior à atual foge do escopo do projeto. 

